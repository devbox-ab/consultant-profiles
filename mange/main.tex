%%%%%%%%%%%%%%%%%%%%%%%%%%%%%%%%%%%%%%%%%
% My CV
%
% This template originates from:
% http://www.LaTeXTemplates.com
%
% Authors:
% Jan Vorisek (jan@vorisek.me)
% Based on a template by Jan Küster (info@jankuester.com)
% Modified for LaTeX Templates by Vel (vel@LaTeXTemplates.com)
%
% License:
% The MIT License (see included LICENSE file)
%
%%%%%%%%%%%%%%%%%%%%%%%%%%%%%%%%%%%%%%%%%

%----------------------------------------------------------------------------------------
%  PACKAGES AND OTHER DOCUMENT CONFIGURATIONS
%----------------------------------------------------------------------------------------

\documentclass[9pt]{template} % Default font size, values from 8-12pt are recommended

%----------------------------------------------------------------------------------------

\begin{document}

%----------------------------------------------------------------------------------------
%  TITLE AND CONTACT INFORMATION
%----------------------------------------------------------------------------------------

\begin{minipage}[t]{0.37\textwidth} % 37% of the page width for name
  \vspace{-\baselineskip} % Required for vertically aligning minipages

  % If your name is very short, use just one of the lines below
  % If your name is very long, reduce the font size or make the minipage wider and reduce the others proportionately
  \colorbox{black}{{\HUGE\textcolor{white}{\textbf{\MakeUppercase{Magnus}}}}} % First name

  \colorbox{black}{{\HUGE\textcolor{white}{\textbf{\MakeUppercase{Bergmark}}}}} % Last name

  \vspace{6pt}

  {\huge Software Craftsman} % Career or current job title
\end{minipage}
\begin{minipage}[t]{0.345\textwidth} % 34.5% of the page width for the first row of icons
  \vspace{-\baselineskip} % Required for vertically aligning minipages

  % The first parameter is the FontAwesome icon name, the second is the box size and the third is the text
  % Other icons can be found by referring to fontawesome.pdf (supplied with the template) and using the word after \fa in the command for the icon you want
  \icon{MapMarker}{12}{Stockholm, Sweden}\\
  \icon{Phone}{12}{+46 70 43 777 42}\\
  \icon{At}{12}{\href{mailto:me@mange.dev}{me@mange.dev}}\\
\end{minipage}
\begin{minipage}[t]{0.285\textwidth} % 28.5% of the page width for the second row of icons
  \vspace{-\baselineskip} % Required for vertically aligning minipages

  % The first parameter is the FontAwesome icon name, the second is the box size and the third is the text
  % Other icons can be found by referring to fontawesome.pdf (supplied with the template) and using the word after \fa in the command for the icon you want
  \icon{Globe}{12}{\href{https://mange.dev}{mange.dev}}\\
  \icon{Github}{12}{\href{https://github.com/Mange}{github.com/Mange}}\\
  \icon{Key}{12}{\href{https://keybase.io/magnusbergmark}{keybase.io/magnusbergmark}}\\
\end{minipage}

\vspace{0.5cm}

%----------------------------------------------------------------------------------------
%  INTRODUCTION, SKILLS AND TECHNOLOGIES
%----------------------------------------------------------------------------------------

\cvsect{Who Am I?}

\begin{minipage}[t]{0.4\textwidth} % 40% of the page width for the introduction text
  \vspace{-\baselineskip} % Required for vertically aligning minipages

  I am a developer with deep interest in building useful solutions to people's
  problems. I have a passion for performant code that is maintainable,
  readable, and fast. There is no perfect tool for every problem, so I try to
  learn as many tools as possible and master them where appropriate.

  You will find me most skilled in backend services and API design, but I'm
  also comfortable working in web frontend stacks. If I'm not able to dogfood
  my own APIs, how can I tell that they deliver value?
\end{minipage}
\hfill % Whitespace between
\begin{minipage}[t]{0.5\textwidth} % 50% of the page for the skills bar chart
  \vspace{-\baselineskip} % Required for vertically aligning minipages
  \textbf{My strongest languages}
  \begin{barchart}{5.5}
    \baritem{Ruby}{100}
    \baritem{HTML/CSS}{70}
    \baritem{JavaScript}{60}
    \baritem{Rust}{60}
    \baritem{Shell}{55}
    \baritem{PostgreSQL}{50}
    \baritem{C}{10}
  \end{barchart}
\end{minipage}

\begin{center}
  \bubbles{5/API design, 6/Backend, 4/Frontend, 3/Infra, 2/Networking}
\end{center}

%----------------------------------------------------------------------------------------
%  Philosophy
%----------------------------------------------------------------------------------------

\cvsect{My development philosophy}

\begin{minipage}[t]{0.4\textwidth} %
  \vspace{-\baselineskip}

  I follow the ethos of the software craftsman, where the tools used are
  adjusted according to the situation at hand. Delivering value is most
  important, then it is making sure that the legacy I leave behind stays
  valuable for a long time. Solutions and code needs to be maintainable and
  understandable by as many people as possible. I try to avoid overly clever
  solutions, and I try to pick conservative tech stacks where appropriate.

  Everything I do is test-driven and I strive for quick feedback cycles, both
  from colleagues and from users.

  In order to deliver software efficiently, one must also work well among other
  people. Coding is not a solitary activity, and being able to work together
  with other people of many different backgrounds and skill levels is
  important. I enjoy working in this condition and I love sharing knowledge and
  understanding other people's perspective on things.
\end{minipage}
\hfill
\begin{minipage}[t]{0.5\textwidth}
  \vspace{-\baselineskip}
  \textbf{Mantras}
  \begin{itemize}
      \item No code is better than a lot of code.
      \item Dependencies are expensive, but custom code can be more expensive.
      \item Explicit is better than implicit.
      \item Code should be working, maintainable, consistent, fast --- in that order.
      \item Testing is not optional (but is so much more than just writing test suites).
      \item Tools should help us. Linters and auto-formatters reduce mental burden.
      \item Real craftsmen sign their work proudly.
      \item ``Legacy'' should not be an ugly word; It's what we all leave behind us.
      \item Most people are well-meaning and knows things I don't know. Don't assume either malice or stupidity.
  \end{itemize}
\end{minipage}

%----------------------------------------------------------------------------------------
%  EXPERIENCE
%----------------------------------------------------------------------------------------

\pagebreak
\cvsect{Experience}

\begin{entrylist}
  \entry
    {Jun 2020 -- Present}
    {Software developer consultant}
    {Devbox}
    {Devbox is a consulting firm. My individual assignments are listed below.}
  \entry
    {Aug 2020 -- Present\\\footnotesize{via Devbox}}
    {Software developer}
    {Health Integrator}
    {Health Integrator is a new startup that focuses on preventive health care
    -- improving people's way of living to avoid later health problems. They
    run programs with Region Stockholm focusing on pre-diabetics, among
    others.\\ I've helped bootstrap this company's technical platform,
    including CI/CD, infra, the main app, and internal development tools. I've
    also helped form the work process with backlog management and similar.\\
      \texttt{Ruby}\slashsep\texttt{Postgres}\slashsep\texttt{Kubernetes}\slashsep\texttt{Google
      Cloud}\slashsep\texttt{Preact}}
  \entry
    {Jun 2020 -- Aug 2020\\\footnotesize{via Devbox}}
    {Software developer}
    {Apoex}
    {Apoex deals with medicine and custom medication manufacturing. While
      working here I was part of the team that worked on software that
      administrated and oversees the manufacturing and shipping of personalized
      medication.\\
      \texttt{Ruby}\slashsep\texttt{Postgres}\slashsep\texttt{MSSQL}\slashsep\texttt{Vue}}
  \entry
    {Mar 2014 -- Jun 2020}
    {Software developer}
    {Hemnet}
    {Hemnet is Sweden's largest property portal and one of the largest websites
      overall, with millions of unique visitors each week. About 1/4 of the
      Swedish population visits Hemnet occasionally.\\ I've been working in the
      Platform team, which deals with internal services, APIs, infrastructure,
      developer tooling, data storage and retrieval, and developer outreach. My
      role over the years has involved setting up better test suites, built
      developer tooling, optimized high-traffic endpoints, secured vulnerable
      endpoints, among other things. I have been designing the API used by
      broker systems to send listing data to us, and was part of the team that
      changed search engine without any downtime or interruption in service.\\
      I've also been working in other teams, shipping user-facing features and
      worked a lot on developer training.\\
      \texttt{Ruby}\slashsep\texttt{Elasticsearch}\slashsep\texttt{Postgres}\slashsep\texttt{GraphQL}\slashsep\texttt{React}\slashsep\texttt{Redis}}
  \entry
    {Jan 2013 -- Mar 2014\\\footnotesize{continued from previous}}
    {Full-stack developer}
    {Bisnode}
    {After a merger with Bisnode, my roles changed. I became more focused on
      teaching agile practices, test-driven development, Ruby/Javascript, and
      setting up runnable business language specifications using
      \texttt{Cucumber}.\\ I also worked a bit with Bisnode's rewrite of an
      older vehicle registration product to a more modern stack with better
      UX and integrating it with their larger Java-based platform. I was called
      in for my expertise in Ruby and test-driven development after they picked
      the stack.\\
      \texttt{Javascript}\slashsep\texttt{Cucumber}\slashsep\texttt{JRuby}}
  \entry
    {Aug 2008 -- Jan 2013}
    {Full-stack developer}
    {Newsline Group}
    {This company was working on search products that indexed most Swedish
      newspaper and online news articles produced, which lead to a extremely
      big database of unstructured text. The company specialized in making this
      huge amount of information searchable and understandable and then tying
      it to structured data about company statements and facts.\\ I was also
      managing the older product portfolio while this new product was built.\\
      Here I managed most of the stack, from Linux server administration and
      provisioning to the CSS used to render articles in the web interface
      after you've found what you were looking for. We built a lot of backend
      software that managed our indices, reindexing and moving around data to
      keep the searches as fast as possible.\\ Newsline Group was later and
      merged with Bisnode.\\
      \texttt{Ruby}\slashsep\texttt{PHP}\slashsep\texttt{Solr}\slashsep\texttt{MySQL}}
\end{entrylist}

%----------------------------------------------------------------------------------------
%  EDUCATION
%----------------------------------------------------------------------------------------

\cvsect{Education}

\begin{entrylist}
  \entry
    {2006 -- 2008}
    {Secondary Education}
    {IT-Gymnasiet Södertörn}
    {I finished my secondary education of a custom program that combined
      software development, web design, and natural sciences. I finished with
      an almost perfect score.}
\end{entrylist}

%----------------------------------------------------------------------------------------
%  OTHER POSITIONS
%----------------------------------------------------------------------------------------

\cvsect{Other positions}

\begin{entrylist}
  \entry
    {2019 -- Present}
    {Scout leader}
    {Vendelsö Scoutkår}
    {I'm currently a Scout leader for my oldest daughter's Scout troop. I've
      taken some courses in leadership in connection with this.}
  \entry
    {2007}
    {Silver medalist}
    {Yrkes-SM Webbdesign}
    {I was competing at the national level in \textit{Yrkes-SM 2007} in the
      category of Web Design and won a silver medal. This competition was
      between teens between 18--19 years old, but I competed anyway despite
      being 17 at the time.}
  \entry
    {2006 -- 2008}
    {Member of IT-council}
    {IT-Gymnasiet Södertörn}
    {I was part of my school's IT-council, which helped to setup the IT
      environment and built some basic tooling regarding registration and
      visualisation of school events.}
\end{entrylist}

%----------------------------------------------------------------------------------------
%  ADDITIONAL INFORMATION
%----------------------------------------------------------------------------------------

\begin{minipage}[t]{0.3\textwidth}
  \vspace{-\baselineskip} % Required for vertically aligning minipages

  \cvsect{Languages}

  \textbf{Swedish} -- native\\
  \textbf{English} -- proficient\\
  \textbf{Latin} -- beginner
\end{minipage}
\hfill
\begin{minipage}[t]{0.3\textwidth}
  \vspace{-\baselineskip} % Required for vertically aligning minipages

  \cvsect{Hobbies}

  I love playing video games, watching movies, reading, and tinkering with
  software. I also have a soft spot for playing Airsoft.\\ If possible, I enjoy
  trying new languages and stacks in my spare time.
\end{minipage}
\hfill
\begin{minipage}[t]{0.3\textwidth}
  \vspace{-\baselineskip} % Required for vertically aligning minipages

  \cvsect{Open source}

  I try to build free software in my spare time and sometimes contribute
  patches to upstream projects. I have contributions accepted into
  \texttt{rails}, \texttt{rspec}, among others. I also have a pretty popular
  Ruby gem called \texttt{roadie} and some Rust libraries.
\end{minipage}

%----------------------------------------------------------------------------------------
%  Tools and skills
%----------------------------------------------------------------------------------------

\cvsect{Tools and skills}

\begin{minipage}[t]{0.3\textwidth} %
  \vspace{-\baselineskip}

  \textbf{Languages}
  \begin{itemize}
      \item C {\footnotesize (entry-level)}
      \item C++ {\footnotesize (entry-level)}
      \item Elm {\footnotesize (entry-level)}
      \item Go {\footnotesize (entry-level)}
      \item Javascript / Typescript {\footnotesize (intermediary/expert)}
      \item PHP {\footnotesize (long time ago)}
      \item Ruby {\footnotesize (expert)}
      \item Rust {\footnotesize (intermediary)}
      \item Shell / Bash {\footnotesize (intermediary)}
  \end{itemize}
\end{minipage}
\hfill
\begin{minipage}[t]{0.3\textwidth}
  \vspace{-\baselineskip}

  \textbf{Methodologies}
  \begin{itemize}
      \item Agile
      \item BDD / TDD
  \end{itemize}

  \textbf{Tools}
  \begin{itemize}
      \item Docker
      \item Git
      \item GraphQL
      \item UNIX systems
      \item Vim
  \end{itemize}
\end{minipage}
\hfill
\begin{minipage}[t]{0.3\textwidth}
  \vspace{-\baselineskip}

  \textbf{Services}
  \begin{itemize}
      \item AWS
      \item GCP
      \item Kubernetes
      \item Elasticsearch
      \item Memcached
      \item PostgreSQL
      \item Redis
  \end{itemize}
\end{minipage}

%----------------------------------------------------------------------------------------

\end{document}
