%%%%%%%%%%%%%%%%%%%%%%%%%%%%%%%%%%%%%%%%%
% My CV
%
% This template originates from:
% http://www.LaTeXTemplates.com
%
% Authors:
% Jan Vorisek (jan@vorisek.me)
% Based on a template by Jan Küster (info@jankuester.com)
% Modified for LaTeX Templates by Vel (vel@LaTeXTemplates.com)
%
% License:
% The MIT License (see included LICENSE file)
%
%%%%%%%%%%%%%%%%%%%%%%%%%%%%%%%%%%%%%%%%%

%----------------------------------------------------------------------------------------
%  PACKAGES AND OTHER DOCUMENT CONFIGURATIONS
%----------------------------------------------------------------------------------------

\documentclass[9pt]{template} % Default font size, values from 8-12pt are recommended

%----------------------------------------------------------------------------------------

\begin{document}

%----------------------------------------------------------------------------------------
%  TITLE AND CONTACT INFORMATION
%----------------------------------------------------------------------------------------

\begin{minipage}[t]{0.52\textwidth} % 45% of the page width for name
  \vspace{-\baselineskip} % Required for vertically aligning minipages

  % If your name is very short, use just one of the lines below
  % If your name is very long, reduce the font size or make the minipage wider and reduce the others proportionately
  \colorbox{black}{{\HUGE\textcolor{white}{\textbf{Stefan Åhman}}}} % Name

  \vspace{6pt}

  {\huge Software Engineer} % Career or current job title
\end{minipage}
\begin{minipage}[t]{0.28\textwidth} % 34.5% of the page width for the first row of icons
  \vspace{-\baselineskip} % Required for vertically aligning minipages

  % The first parameter is the FontAwesome icon name, the second is the box size and the third is the text
  % Other icons can be found by referring to fontawesome.pdf (supplied with the template) and using the word after \fa in the command for the icon you want
  \icon{MapMarker}{12}{Stockholm, Sweden}\\
  \icon{Phone}{12}{+46 (0)70 616 85 70}\\
  \icon{At}{12}{\href{mailto:stefan@devbox.com}{stefan@devbox.com}}\\
\end{minipage}
\begin{minipage}[t]{0.285\textwidth} % 28.5% of the page width for the second row of icons
  \vspace{-\baselineskip} % Required for vertically aligning minipages

  % The first parameter is the FontAwesome icon name, the second is the box size and the third is the text
  % Other icons can be found by referring to fontawesome.pdf (supplied with the template) and using the word after \fa in the command for the icon you want
  % \icon{Globe}{12}{\href{https://example.com}{example.com}}\\
  \icon{Github}{12}{\href{https://github.com/stefanahman}{github.com/stefanahman}}\\
  \icon{Key}{12}{\href{https://keybase.io/stefanahman}{keybase.io/stefanahman}}\\
\end{minipage}

\vspace{0.5cm}

%----------------------------------------------------------------------------------------
%  INTRODUCTION, SKILLS AND TECHNOLOGIES
%----------------------------------------------------------------------------------------

\cvsect{Who Am I?}

\begin{minipage}[t]{0.4\textwidth} % 40% of the page width for the introduction text
  \vspace{-\baselineskip} % Required for vertically aligning minipages
  I value smaller and open-minded, less bureaucratic companies that are willing to try new ideas and be outside the box.\\

  A long term vision I have is to be an inspiration to as many people as possible, and to build an awesome company.\\

  Besides riding my motorcycle, I enjoy traveling, hiking, running, and taking care of my plants. My most recent interests are ashtanga yoga and contract bridge.
\end{minipage}
\hfill % Whitespace between
\begin{minipage}[t]{0.5\textwidth} % 50% of the page for the skills bar chart
  \vspace{-\baselineskip} % Required for vertically aligning minipages
  \begin{barchart}{5.0}
    \baritem{Ruby On Rails}{100}
    \baritem{HTML/CSS}{80}
    \baritem{JavaScript}{70}
    \baritem{Elixir}{30}
    \baritem{PostgreSQL}{80}
    \baritem{MySQL}{60}
    \baritem{SQL Server}{20}
    \baritem{Google Cloud Platform}{90}
    \baritem{Kubernetes/Terraform}{70}
    \baritem{AWS/Microsoft Azure}{40}
  \end{barchart}
\end{minipage}


%----------------------------------------------------------------------------------------
%  Philosophy
%----------------------------------------------------------------------------------------

\cvsect{My development philosophy}

\begin{minipage}[t]{0.4\textwidth} %
  \vspace{-\baselineskip}

  Mind of creating great code and efficient but robust solutions to realize great visions.\\

  Development is not about working in the office from 8 to 5. It's about creating beautiful art in places and times when you are the most creative and inspired.

  Tests are just as important as the code to build sustainable solutions with confidence.\\

  It's better to focus on fewer things at the same time and make them superb, preferably with continuous deployments.\\

  Developer happiness is more important than you think, without a happy environment, it's tough to create good solutions.
\end{minipage}
\hfill
\begin{minipage}[t]{0.5\textwidth}
  \vspace{-\baselineskip}
  \textbf{Mantras}
  \begin{itemize}
      \item Enjoy your life, always to the fullest
      \item Delivering value over getting caught in the details
      \item Quality over shortcuts (with exceptions)
      \item Do stuff you can be proud of
      \item Static office hours are obsolete
      \item The key to success is transparency and honesty
      \item A functional team is more important than an idea
      \item Strongly considering making \texttt{git dad} an alias
  \end{itemize}
\end{minipage}

%----------------------------------------------------------------------------------------
%  EXPERIENCE
%----------------------------------------------------------------------------------------

\pagebreak
\cvsect{Experience}

\begin{entrylist}
  \entry
    {June 2022 -- Present}
    {Software Engineer}
    {Devbox}
    {Full time employed software engineer consultant. It's time to
      explore new environments, meet people and grow within the software
      industry.
      %My individual assignments are listed below.
      }
  % \entry
  %   {Sept 2022 -- Present\\\footnotesize{via Devbox}}
  %   {Software Engineer}
  %   {TBD}
  %   {TBD.\\
  %     \texttt{TBD}\slashsep\texttt{TBD}\slashsep\texttt{TBD}}
  \entry
    {April 2021 -- Present}
    {Software Engineer Consultant}
    {ICANIWILL}
    {Developing a platform for internal purposes.\\
      \texttt{Ruby on Rails}\slashsep
      \texttt{Javascript}\slashsep
      \texttt{Microsoft Azure}
    }
  \entry
    {Jan 2022 -- June 2022}
    {Personal growth}
    {South East Asia}
    {I decided to take some time off after my last adventure, mainly focusing on
      personal development, but also figuring out what to do next. Leaving Sweden
      by myself for the first time was a big thing for me, heading off to
      Cambodia as my first destination. Without a further plan and literally
      living day by day, discovering new food, exploring historical landmarks
      and meeting people with different cultural backgrounds was so inspiring.\\
      At first, I felt a little lost in this new and unknown environment. As I
      reached Chiang Mai, Thailand, I'm immediately starting to feel like home and
      very much fell in love with the country, with the street food, night markets,
      nature, local people. After attending my very first yoga retreat, I
      gained more confidence, discovering mantras, and finding new habits, some
      of them still with up until this day. What really had an impact on me was
      the digital detox and how our devices, primarily social media affects.
      At this point, I found out that I wanted to go a way to get of screen a
      bit more and interacting with people and feelings.\\
      \texttt{Nature}\slashsep
      \texttt{Smoothies}\slashsep
      \texttt{Socialising}\slashsep
      \texttt{Street food}\slashsep
      \texttt{Scuba diving}\slashsep
      \texttt{Yoga}
    }
  \entry
    {Sept 2018 - Jan 2022}
    {Tech Lead}
    {ApoEx}
    {Led the technical vision within the team and support developers in
      technical decision-making. Making sure the team can develop with
      confidence thanks to the review process and the solid test suite and
      deployment pipeline. Developing a guide how to write maintainable
      and testable code using the internal style guidelines.\\
      Responsible for migrating multiple full scale applications to Google Cloud
      and Kubernetes from an On-Prem Apache setup.\\
      Managed and acted as a stakeholder for the technical backlog in my team.\\
      A migration from Resque to Sidekiq were in place to optimize performance
      on background jobs.\\
      \texttt{Google Cloud Plattform}\slashsep
      \texttt{Terraform}\slashsep
      \texttt{Sidekiq}
    }
  \entry
    {Sept 2014 - Jan 2022}
    {Full Stack Developer}
    {ApoEx}
    {Lead developer for a multi-site platform handling cytostatic/sterile
      extempore compounding within Sweden.
      Lead developer for a multi-site platform handling dose dispensing within
      inpatient care within Sweden.\\
      Integrating with regions all across the country supporting multiple
      journal systems allowing feature toggling different complex independent
      features including order management, logistics, data analytics and
      invoicing as well as managing remainders.\\
      We were taking a big role in leading the way and designing the
      standardization of the national electronic attachment used by all
      pharmacies in the industry, and that would later be important for the
      future.\\
      Fully integrated third-party systems with Web Service and XML/CSV file
      interaction were used to establish full support for the
      production line.\\
      With features such as goods receipts, pharmaceutical controlling, delivery
      package sorting and shipping, we spent a lot of time developing the UX to
      be as uniformed and intuitive as possible. Leading forums with the
      customers to continuously improving it.\\
      Background jobs crunching data including processing orders, updating stock
      balances and generating invoice bases, booking transports, and much more,
      were monitored and optimized when needed.\\
      Using a big data database to aggregate production data to easier create
      queries for internal and external statistics reports. The data were
      continuously appended using Google Cloud Platform message queue, PubSub.\\
      Another application using serverless technology were used to match goods
      receipts with invoices bases every month to ensure legit invoices from
      suppliers containing many millions of lines every month.\\
      \texttt{Ruby on Rails}\slashsep
      \texttt{Javascript}\slashsep
      \texttt{HTML/CSS}\slashsep
      \texttt{PostgreSQL}\slashsep
      \texttt{Redis}\slashsep
      \texttt{Elixir}\slashsep
      \texttt{RabbitMQ}\slashsep
      \texttt{POSIX scripting}\slashsep
      \texttt{WebServices (XML)}\slashsep
      \texttt{SQL Server}
    }
  \entry
    {Feb 2012 - Sept 2014}
    {Android Developer}
    {AudioApps}
    {We developed an audio guide for museums and exhibitions. By using QR codes,
      people could listen to a audio description of the piece of art in front
      of them.\\
      \texttt{Java}\slashsep
      \texttt{Android SDK}
    }
\end{entrylist}

\pagebreak

%----------------------------------------------------------------------------------------
%  EDUCATION
%----------------------------------------------------------------------------------------

\cvsect{Education}

\begin{entrylist}
  \entry
    {2009 - 2014}
    {Postgraduate Education}
    {KTH Royal Institute of Technology}
    {Degree Program in Information and Communication Technology, Master's
      Program, Software Engineering of Distributed Systems.}
\end{entrylist}

%----------------------------------------------------------------------------------------
%  ADDITIONAL INFORMATION
%----------------------------------------------------------------------------------------

\begin{minipage}[t]{0.3\textwidth}
  \vspace{-\baselineskip} % Required for vertically aligning minipages

  \cvsect{Languages}

  \textbf{Swedish} - native\\
  \textbf{English} - proficient
\end{minipage}
\hfill
\begin{minipage}[t]{0.7\textwidth}
  \vspace{-\baselineskip} % Required for vertically aligning minipages

  \cvsect{Hobbies}

  Golf, Padel, Running, Motorcycling, Nature, Trekking, Cooking, Yoga
\end{minipage}

%----------------------------------------------------------------------------------------
%  Tools and skills
%----------------------------------------------------------------------------------------

\cvsect{Tools and skills}

\begin{minipage}[t]{0.3\textwidth} %
  \vspace{-\baselineskip}

  \textbf{Languages}
  \begin{itemize}
      \item Ruby on Rails
      \item Javascript
      \item Terraform
      \item Go
      \item Elixir
      \item POSIX scripting
  \end{itemize}
\end{minipage}
\hfill
\begin{minipage}[t]{0.3\textwidth}
  \vspace{-\baselineskip}

  \textbf{Methodologies}
  \begin{itemize}
      \item Agile
      \item TDD
  \end{itemize}

  \textbf{Tools}
  \begin{itemize}
      \item Docker
      \item Git
      \item UNIX systems
      \item Vim
  \end{itemize}
\end{minipage}
\hfill
\begin{minipage}[t]{0.3\textwidth}
  \vspace{-\baselineskip}

  \textbf{Services}
  \begin{itemize}
      \item Google Cloud Plattform
      \item Amazon Web Services
      \item Microsoft Azure
      \item Kubernetes
      \item RabbitMQ
      \item PostgreSQL
      \item Redis
  \end{itemize}
\end{minipage}

%----------------------------------------------------------------------------------------

\end{document}
