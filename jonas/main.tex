%%%%%%%%%%%%%%%%%%%%%%%%%%%%%%%%%%%%%%%%%
% My CV
%
% This template originates from:
% http://www.LaTeXTemplates.com
%
% Authors:
% Jan Vorisek (jan@vorisek.me)
% Based on a template by Jan Küster (info@jankuester.com)
% Modified for LaTeX Templates by Vel (vel@LaTeXTemplates.com)
%
% License:
% The MIT License (see included LICENSE file)
%
%%%%%%%%%%%%%%%%%%%%%%%%%%%%%%%%%%%%%%%%%

%----------------------------------------------------------------------------------------
%  PACKAGES AND OTHER DOCUMENT CONFIGURATIONS
%----------------------------------------------------------------------------------------

\documentclass[9pt]{template} % Default font size, values from 8-12pt are recommended

%----------------------------------------------------------------------------------------

\begin{document}

%----------------------------------------------------------------------------------------
%  TITLE AND CONTACT INFORMATION
%----------------------------------------------------------------------------------------

\begin{minipage}[t]{0.45\textwidth} % 45% of the page width for name
  \vspace{-\baselineskip} % Required for vertically aligning minipages

  % If your name is very short, use just one of the lines below
  % If your name is very long, reduce the font size or make the minipage wider and reduce the others proportionately
  \colorbox{black}{{\HUGE\textcolor{white}{\textbf{\MakeUppercase{Jonas}}}}} % First name

  \colorbox{black}{{\HUGE\textcolor{white}{\textbf{\MakeUppercase{Liljestrand}}}}} % Last name

  \vspace{6pt}

  {\huge Software Engineer} % Career or current job title
\end{minipage}
\begin{minipage}[t]{0.345\textwidth} % 34.5% of the page width for the first row of icons
  \vspace{-\baselineskip} % Required for vertically aligning minipages

  % The first parameter is the FontAwesome icon name, the second is the box size and the third is the text
  % Other icons can be found by referring to fontawesome.pdf (supplied with the template) and using the word after \fa in the command for the icon you want
  \icon{MapMarker}{12}{Stockholm, Sweden}\\
  \icon{Phone}{12}{+46 73 99 397 56}\\
  \icon{At}{12}{\href{mailto:jonas.liljestrand@gmail.com}{jonas.liljestrand@gmail.com}}\\
\end{minipage}
\begin{minipage}[t]{0.285\textwidth} % 28.5% of the page width for the second row of icons
  \vspace{-\baselineskip} % Required for vertically aligning minipages

  % The first parameter is the FontAwesome icon name, the second is the box size and the third is the text
  % Other icons can be found by referring to fontawesome.pdf (supplied with the template) and using the word after \fa in the command for the icon you want
  % \icon{Globe}{12}{\href{https://example.com}{example.com}}\\
  \icon{Github}{12}{\href{https://github.com/jonlil}{github.com/jonlil}}\\
  \icon{Key}{12}{\href{https://keybase.io/jonasliljestrand}{keybase.io/jonasliljestrand}}\\
\end{minipage}

\vspace{0.5cm}

%----------------------------------------------------------------------------------------
%  INTRODUCTION, SKILLS AND TECHNOLOGIES
%----------------------------------------------------------------------------------------

\cvsect{Who Am I?}

\begin{minipage}[t]{0.4\textwidth} % 40% of the page width for the introduction text
  \vspace{-\baselineskip} % Required for vertically aligning minipages

  I am a developer with interest in building useful solutions. I have a passion
  for learning new things that can help me solve upcoming problems.
  I think collaborative work is awesome and really appreciate creating
  solutions as a team.

  You will find me most skilled in managing cloud infrastructure, but I'm also
  comfortable working with test driven software development. It is also possible
  finding me decoding binary protocols in WireShark.
\end{minipage}
\hfill % Whitespace between
\begin{minipage}[t]{0.5\textwidth} % 50% of the page for the skills bar chart
  \vspace{-\baselineskip} % Required for vertically aligning minipages
  \textbf{My strongest languages}
  \begin{barchart}{5.5}
    \baritem{Terraform}{100}
    \baritem{Ruby}{90}
    \baritem{Python}{80}
    \baritem{JavaScript}{80}
    \baritem{Golang}{70}
    \baritem{Rust}{70}
    \baritem{HTML/CSS}{70}
    \baritem{Shell}{70}
    \baritem{PostgreSQL}{50}
  \end{barchart}
\end{minipage}

\begin{center}
  \bubbles{4/API design, 5/Backend, 4/Frontend, 6/Infra, 4/Networking}
\end{center}

%----------------------------------------------------------------------------------------
%  Philosophy
%----------------------------------------------------------------------------------------

\cvsect{My development philosophy}

\begin{minipage}[t]{0.4\textwidth} %
  \vspace{-\baselineskip}

  I follow the ethos of the software craftsman, where the tools used are
  adjusted according to the situation at hand. Delivering value is most
  important, then it is making sure that the legacy I leave behind stays
  valuable for a long time. Solutions and code needs to be maintainable and
  understandable by as many people as possible. I try to avoid overly clever
  solutions, and I try to pick conservative tech stacks where appropriate.

  Everything I do is test-driven and I strive for quick feedback cycles, both
  from colleagues and from users.

  In order to deliver software efficiently, one must also work well among other
  people. Coding is not a solitary activity, and being able to work together
  with other people of many different backgrounds and skill levels is
  important. I enjoy working in this condition and I love sharing knowledge and
  understanding other people's perspective on things.
\end{minipage}
\hfill
\begin{minipage}[t]{0.5\textwidth}
  \vspace{-\baselineskip}
  \textbf{Mantras}
  \begin{itemize}
      \item No code is better than a lot of code.
      \item Dependencies are expensive, but custom code can be more expensive.
      \item Explicit is better than implicit.
      \item Code should be working, maintainable, consistent, fast --- in that order.
      \item Testing is not optional (but is so much more than just writing test suites).
      \item Tools should help us. Linters and auto-formatters reduce mental burden.
      \item Real craftsmen sign their work proudly.
      \item ``Legacy'' should not be an ugly word; It's what we all leave behind us.
      \item Most people are well-meaning and knows things I don't know. Don't assume either malice or stupidity.
      \item Measure twice and cut once.
  \end{itemize}
\end{minipage}

%----------------------------------------------------------------------------------------
%  EXPERIENCE
%----------------------------------------------------------------------------------------

\pagebreak
\cvsect{Experience}

\begin{entrylist}
  \entry
    {Sept 2021 -- Present}
    {Software developer consultant}
    {Devbox}
    {Devbox is a consulting firm. My individual assignments are listed below.}
  \entry
    {Sept 2021 -- Present\\\footnotesize{via Devbox}}
    {Software Developer}
    {Joint Academy}
    {Joint Academy connects patients with licensed physical therapists to deliver
      online treatments for chronic joint and back pain.\\
      I was part of the platform team helping design a microservice infrastructure
      to support future growth. My role involved moving the entire infrastructure
      to AWS, built a new authentication service utilizing BankID and helped improve
      codebase test coverage.\\
      \texttt{Ruby}\slashsep\texttt{AWS}\slashsep\texttt{Docker}\slashsep\texttt{TypeScript}\slashsep\texttt{Postgres}}
  \entry
    {Mar 2017 -- Sept 2021}
    {Software Developer}
    {Hemnet}
    {Hemnet is Sweden's largest property portal and one of the largest websites
      overall, with millions of unique visitors each week. About 1/4 of the
      Swedish population visits Hemnet occasionally.\\ I've been working in the
      Platform team, which deals with internal services, APIs, infrastructure,
      developer tooling, data storage and retrieval, and developer outreach. My
      role over the years has involved migrating Hemnet to \texttt{AWS}.
      As part of that I built CI/CD pipelines in \texttt{Jenkins}
      that build, tested and shipped \texttt{Docker} images.
      Also acted as the developer sales department could bring to external meetings
      when technical expertise was required.\\
      \texttt{Ruby}\slashsep\texttt{AWS}\slashsep\texttt{Docker}\slashsep\texttt{Terraform}\slashsep\texttt{Jenkins}\slashsep\texttt{Postgres}}
  \entry
    {May 2015 -- Mar 2017}
    {Full-stack developer}
    {Get a Newsletter Scandinavia AB}
    {Get a Newsletter is a marketing tool for creating and sending of newsletters
      per email. The majority of the time I was there I worked as a single developer
      managing all parts of the system. That included configuring \texttt{PowerMTA}
      for sending millions of emails per day. I was hired as a frontend developer
      since I had experience with \texttt{AngularJS}.\\ I built and shipped code
      that was used in the tool for creating the newsletters. During this time I
      had to learn \texttt{Puppet} in order to maintain the infrastructure.
      I also worked with rewriting a \texttt{Python} backend service for improving
      concurrency of generating emails to be able to handle more daily customers.
      For this I introduced \texttt{RabbitMQ} which was a very good fit.
      The migration for that was really exciting and was handled without downtime
      for end users.\\
      \texttt{Javascript}\slashsep\texttt{AngularJS}\slashsep\texttt{PowerMTA}\slashsep\texttt{Puppet}\slashsep\texttt{Python}}
  \entry
    {Sept 2014 -- May 2015}
    {Mobile/Web developer}
    {Vimy}
    {Vimy is a job search site that was called YouLinker when I worked there.
      I was part of a team that worked on a solution for replacing the paper CV
      with recorded video presentation. The solution worked for some professions like
      sales. The application was written in a framework called \texttt{Ionic} that
      used \texttt{Cardova} to cross compile into native mobile applications.\\
      \texttt{AngularJS}\slashsep\texttt{Ionic}\slashsep\texttt{Cordova}\slashsep\texttt{HTML/CSS}}
  \entry
    {Apr 2014 -- May 2015\\\footnotesize{continued from previous}}
    {Software Developer/Co-Founder}
    {Give AB}
    {Creddly team pivoted and started creating Give. The social gifting app where people
      could buy and send coupons via SMS or social media to friends or family.
      The coupon could then be used in a physical store to redeem the gift.
      It was an awesome feeling building this and it still makes me happy thinking of the
      first time we went to the store next door and redeemed our first ice-cream.
      Later the focus shifted against business to business.
      I built the \texttt{single page application} in \texttt{AngularJS} where customers
      could manage their gifting campaigns. I was also responsible for writing the API
      the frontend consumed which gave me a lot of experience with \texttt{MongoDB} and
      \texttt{NodeJS}. I also built and designed the API external partners could
      integrate Give in their workflows for automating gift delivery. Give was acquired
      by an other company in 2017.
      \texttt{NodeJS}\slashsep\texttt{MongoDB}\slashsep\texttt{AngularJS}\slashsep\texttt{API Design}}
  \entry
    {Apr 2013 - Apr 2014}
    {Software Developer}
    {Creddly AB}
    {Creddly was a social shopping app for both iOS and Android. I was working on building
      a backoffice web application for managing the products available in the system.
      The backoffice application had a frontend and backend which I designed myself.\\
      \texttt{NodeJS}\slashsep\texttt{MongoDB}\slashsep\texttt{AngularJS}\slashsep\texttt{API Design}}
  \entry
    {Sept 2011 - Apr 2013}
    {Software Developer}
    {Return Great AB}
    {Consulting firm with the idea that we iterated and improved our clients
      business ideas, developed them and returned a great product. We were
      also in charge of ComHem and Mekonomens’s web based employee motivation
      systems. These were quite old and outdated systems written in \texttt{ASP.NET}
      with a \texttt{MSSQL} database.\\
      \texttt{ASP.NET}\slashsep\texttt{MSSQL}}
\end{entrylist}

\pagebreak

%----------------------------------------------------------------------------------------
%  EDUCATION
%----------------------------------------------------------------------------------------

\cvsect{Education}

\begin{entrylist}
  \entry
    {2003 -- 2005}
    {Secondary Education}
    {Vasagymnasiet Arboga}
    {I finished my secondary education of a custom program that combined
      software development, programming, and natural sciences.}
\end{entrylist}

%----------------------------------------------------------------------------------------
%  OTHER POSITIONS
%----------------------------------------------------------------------------------------

\cvsect{Other positions}

\begin{entrylist}
  \entry
    {2020 -- Present}
    {Materials Manager}
    {Nacka Hockey - Team 13}
    {I'm currently a materials manager for my sons hockey team. Where I'm
      responsible for having the equipment need. Also spend some time sharpening
      skates. This is really funny and good practice for leadership.}
\end{entrylist}

%----------------------------------------------------------------------------------------
%  ADDITIONAL INFORMATION
%----------------------------------------------------------------------------------------

\begin{minipage}[t]{0.3\textwidth}
  \vspace{-\baselineskip} % Required for vertically aligning minipages

  \cvsect{Languages}

  \textbf{Swedish} - native\\
  \textbf{English} - proficient
\end{minipage}
\hfill
\begin{minipage}[t]{0.7\textwidth}
  \vspace{-\baselineskip} % Required for vertically aligning minipages

  \cvsect{Hobbies}

  I love playing music at parties and has been a part-time DJ for over 10 years,
  playing at private parties and nightclubs. I really like the combination of seeing
  people happy and being a bit nerdy with all audio and DJ equipment. I have also
  performed some live streaming and really enjoy creating interesting video angles.

  I love playing video games, watching moves and playing around with both hardware
  and software.

  I also like outdoor activities like skiing, hiking. And I'm playing veteran
  ice hockey, soccer and padel and like being healthy.
\end{minipage}

%----------------------------------------------------------------------------------------
%  Tools and skills
%----------------------------------------------------------------------------------------

\cvsect{Tools and skills}

\begin{minipage}[t]{0.3\textwidth} %
  \vspace{-\baselineskip}

  \textbf{Languages}
  \begin{itemize}
      \item Terraform {\footnotesize (expert)}
      \item Javascript / Typescript {\footnotesize (expert)}
      \item Ruby {\footnotesize (intermediary)}
      \item Python {\footnotesize (intermediary)}
      \item Rust {\footnotesize (intermediary)}
      \item Shell / Bash {\footnotesize (intermediary/expert)}
      \item Go {\footnotesize (intermediary)}
      \item Groovy {\footnotesize (entry-level)}
      \item PHP {\footnotesize (long time ago)}
      \item Puppet {\footnotesize (intermediary/expert)}
  \end{itemize}
\end{minipage}
\hfill
\begin{minipage}[t]{0.3\textwidth}
  \vspace{-\baselineskip}

  \textbf{Methodologies}
  \begin{itemize}
      \item Agile
      \item BDD / TDD
  \end{itemize}

  \textbf{Tools}
  \begin{itemize}
      \item Docker
      \item Git
      \item UNIX systems
      \item Vim
  \end{itemize}
\end{minipage}
\hfill
\begin{minipage}[t]{0.3\textwidth}
  \vspace{-\baselineskip}

  \textbf{Services}
  \begin{itemize}
      \item AWS
      \item Elasticsearch
      \item PowerMTA
      \item RabbitMQ
      \item Jenkins
      \item Memcached
      \item PostgreSQL
      \item Redis
      \item Logstash
      \item Kibana
      \item NGiNX
      \item Locust
  \end{itemize}
\end{minipage}

%----------------------------------------------------------------------------------------

\end{document}
